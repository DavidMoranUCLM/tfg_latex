\documentclass{sty/eiiatfg}

%\setmonofont{Noto Sans Mono}

\usepackage[nomain,acronym]{glossaries}
\makeglossaries

% Hay muchos paquetes para LaTeX que facilitan centenares de trabajos
% engorrosos. Busca en Internet si no sabes hacer algo con LaTeX. El
% repositorio oficial está en https://ctan.org/

% Algunos paquetes especialmente útiles ya están incluídos en el estilo
% del TFG (consulta sty/eiiatfg.cls para más detalles).

% Cambia los datos de tu TFG en este archivo

\title{Plantilla \LaTeX{} para elaborar el TFG en la Escuela de Ingeniería Industrial y Aeroespacial}
\author{David Morán Soriano}

% IE = Ingeniería Eléctrica
% IEIA = Ingeniería Electrónica Industrial y Automática
% IA = Ingeniería Aeroespacial
\grado{IE}

% Tu número de expediente puedes consultarlo en secretaría virtual
\expediente{225655}

% En caso de múltiples directores separa los nombres con \\
% Si solo hay un tutor no pongas \\
\tutor{%
    Francisco Moya Fernández\\
    Fernando Castillo García}


% A partir de aquí se trata de datos opcionales. Si algún dato no quieres que figure borra la línea o coméntala poniendo el signo % al principio

% Es muy conveniente proporcionar un medio de contacto con el autor.  El correo electrónico es probablemente el menos invasivo. No uses correos con nicknames extraños, es un documento profesional. En caso de duda usa el correo de la Universidad, pero recuerda que dejará de ser válido unas semanas después de que dejes de ser alumno.
\email{david.moran@alu.uclm.es}

% No uses tu teléfono personal, será accesible para cualquier usuario de la biblioteca
\phone{925 268 800}

% Una página web para el proyecto puede ser un requisito necesario en caso de que sea trabajo parcialmente financiado con un proyecto de I+D. Consulta a tu tutor
\homepage{https://github.com/UCLM-eiia-to/eiia_doc_tfg_latex}

% Tener un repositorio GIT (http://github.com) permite llevar un control de versiones. Tu tutor puede considerarlo esencial, habla con él
\gitrepo{https://github.com/UCLM-eiia-to/eiia_doc_tfg_latex.git}

% Pon una dirección si puede ser interesante para recibir correspondencia relacionada. No pongas tu dirección personal
\address{UCLM --- Escuela de Ingeniería Industrial y Aeroespacial\\
    Campus Universitario de la Real Fábrica de Armas}
\poblacion{Toledo}
\cpostal{45004}

\input{tex/acronimos.tex}

% La bibliografía la puedes descomponer en varios archivos .bib
% Los archivos .bib se pueden escribir a mano con ayuda de un editor online
% (e.g. http://truben.no/latex/bibtex) o generar con Mendeley u otro
% gestor de bibliografía. Solo se incluyen las referencias que son citadas
% en el texto.

\addbibresource{bib/main.bib}
\addbibresource{bib/how.bib}
\addbibresource{bib/ejemplos.bib}

\begin{document}

% Puedes cambiar la licencia de este documento con la orden license.  Por defecto se asume Creative Commons Attribution 4.0 (ver https://creativecommons.org/licenses/by/4.0/).  
% Por ejemplo, para restringir su uso, copia y distribución:
%\license{Todos los derechos reservados.}

% Si usas muchos símbolos conviene que describas lo que significan en un archivo
% aparte (en este caso simbolos.tex).  Si no es el caso puedes comentar esta línea
\listofsymbols{tex/simbolos.tex}


\portada

% Edita los
\input{tex/agradecimientos.tex}	   
\input{tex/dedicatoria.tex}
\input{tex/resumen.tex}

\indices

% El cuerpo del documento está en la carpeta tex
% Aquí simplemente se incluyen los archivos correspondientes a cada capítulo.

% No los llames capitulo1, capitulo2, etc. Los números los pondrá LaTeX según
% el orden en que los pongas.  Eso facilitará después su posible reordenación
% o división.

% Esta estructura corresponde a un documento científico-técnico.

% Si ves que tu proyecto concuerda con un trabajo profesional organiza el 
% documento según UNE 157001

% Si ves que tu proyecto se explica mejor en otro orden o con otra organización 
% hazlo sin preguntar.  No hay un modelo impuesto por la normativa

\input{tex/introduccion.tex}
\input{tex/objetivos.tex}
\input{tex/antecedentes.tex}
\input{tex/desarrollo.tex}
\input{tex/resultados.tex}
\input{tex/conclusiones.tex}

% Los anexos pueden ir en la misma carpeta que el cuerpo del documento 
% porque eso facilita la migración de partes de un capítulo a un anexo.
\appendix\cleardoublepage
\hypertarget{ch:anexos}{%
    \input{tex/latex.tex}}

% La bibliografía no suele ir numerada porque se pone después de los anexos.
% No se debe poner antes de los anexos porque si se cita una referencia en
% un anexo sería una backward reference, que deben evitarse a toda costa
\cleardoublepage
\hypertarget{ch:bibliografia}{%
    \printbibliography[heading=bibintoc]}
\cleardoublepage

\end{document}
