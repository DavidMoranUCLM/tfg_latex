%\documentclass{sty/eiiatfg}
\documentclass{sty/propuesta}
%\setmonofont{Noto Sans Mono}

\addbibresource{bib/main.bib}

%% Cambia los datos de tu TFG en este archivo

\title{Plantilla \LaTeX{} para elaborar el TFG en la Escuela de Ingeniería Industrial y Aeroespacial}
\author{David Morán Soriano}

% IE = Ingeniería Eléctrica
% IEIA = Ingeniería Electrónica Industrial y Automática
% IA = Ingeniería Aeroespacial
\grado{IE}

% Tu número de expediente puedes consultarlo en secretaría virtual
\expediente{225655}

% En caso de múltiples directores separa los nombres con \\
% Si solo hay un tutor no pongas \\
\tutor{%
    Francisco Moya Fernández\\
    Fernando Castillo García}


% A partir de aquí se trata de datos opcionales. Si algún dato no quieres que figure borra la línea o coméntala poniendo el signo % al principio

% Es muy conveniente proporcionar un medio de contacto con el autor.  El correo electrónico es probablemente el menos invasivo. No uses correos con nicknames extraños, es un documento profesional. En caso de duda usa el correo de la Universidad, pero recuerda que dejará de ser válido unas semanas después de que dejes de ser alumno.
\email{david.moran@alu.uclm.es}

% No uses tu teléfono personal, será accesible para cualquier usuario de la biblioteca
\phone{925 268 800}

% Una página web para el proyecto puede ser un requisito necesario en caso de que sea trabajo parcialmente financiado con un proyecto de I+D. Consulta a tu tutor
\homepage{https://github.com/UCLM-eiia-to/eiia_doc_tfg_latex}

% Tener un repositorio GIT (http://github.com) permite llevar un control de versiones. Tu tutor puede considerarlo esencial, habla con él
\gitrepo{https://github.com/UCLM-eiia-to/eiia_doc_tfg_latex.git}

% Pon una dirección si puede ser interesante para recibir correspondencia relacionada. No pongas tu dirección personal
\address{UCLM --- Escuela de Ingeniería Industrial y Aeroespacial\\
    Campus Universitario de la Real Fábrica de Armas}
\poblacion{Toledo}
\cpostal{45004}


\title{Propuesta de TFG}
\author{David Morán Soriano}

\begin{document}

\maketitle
\tableofcontents
\newpage

\section{Introducción}
En este documento se recogerá la propuesta inicial de un TFG para el grado de \emph{Ingeniería Aeroespacial} en la \emph{UCLM}.

Para este TFG se plantea el estudio, desarrollo e implementación de un sistema de navegación aérea de bajo coste mediante el empleo de medidas inerciales tomadas por sensores MEMS y respaldadas por posicionamiento satelital y medidas barométricas.

Con el fin de unificar toda la información adquirida por los distintos sensores se propone emplear un conjunto de Filtros de Kalman Expandidos ejecutándose en un SoC de bajo presupuesto.


\section{Objetivos}
A continuación se listarán los objetivos prelimirares principales y secundarios que se pretende explorar durante el desarrollo del TFG.
\begin{itemize}
    \item Principales
    \begin{itemize}
        \item Lectura en bruto de los sensores de forma periódica y confiable.
        \item Preprocesado e interpretación de lecturas.
        \item Registrar medidas de forma local(Tarjeta SD).
        \item Creación de un modelo de filtro de Kalman ajustado al sistema dinámico con el que se trabaja.
        \item Modelado de un sistema dinámico que permita probar el filtro de Kalman.
        \item Integración del filtro de Kalman y lectura de sensores en el hardware escogido.
        \item Evaluación de rendimiento del sistema (Latencia + precisión + consumo).
    \end{itemize}
    \item Secundarios
    \begin{itemize}
        \item Grabar rutas de navegación en Tajeta SD.
        \item Comparar ruta respecto a medidas.
        \item Añadir telemetría.
        \item Proponer correcciones.
        \item Actuar conforme a las correcciones.
    \end{itemize}
\end{itemize}

\section{Planificación}

Para abordar este proyecto se plantea tomar un enfoque expansivo en el que se comience alcanzando unos objetivos mínimos y, posteriormente se realice un proceso iterativo en el que se vaya ampliando y mejorando la funcionalidad del sistema.

Este enfoque requiere que se realice un estudio inicial en el que se plantee el alcance máximo del proyecto, de forma que se pueda diseñar el código para ser capaz de incorporar nueva funcionalidad sin necesidad de realizar alteraciones significativas.

Además, dado que se pretende alterar continuamente el código, serán necesarios mecanismos que permitan facilitar y automatizar la verificación de la funcionalidad de forma retroactiva.

El TFG estará dividido en 3 secciones principales, que serán las que se irán ejecutando de forma iterativa.

La primera consistirá en una fase de investigación en la que se estudie y plantee el modelo que se quiera implementar. El resultado será un modelo matemático y una propuesta breve sobre como implementarlo.

La segunda traducirá el modelo a un lenguaje de programación, comúnmente C, aunque también se podrán emplear Python y Matlab para hacer pruebas de funcionalidad o hacer pruebas más rápidamente. Como resultado se obtendrá un código que se podrá ejecutar en el hardware escogido.

Finalmente, se realizará una fase de evaluación en la que se comprobará el desempeño del modelo en un caso controlado, en el que se pueda comparar los resultados con los obtenidos, tanto en iteraciones anteriores como respecto a las especificaciones planteadas.

A continuación se desarrollará el proceso de cada fase preliminarmente para la primera iteración.

\subsection{Investigación}

Durante esta fase se estudiará los métodos físicos y matemáticos que describen a nuestro sistema y se seleccionarán los más óptimos. También se tratará de investigar algoritmos que puedan acelerar la ejecución de código.

El proceso de investigación se realizará de forma incremental siguiendo el siguiente desarrollo:

\begin{enumerate}
    \item Establecer sistemas de referencia ground y body. \cite[sec.~1.2 - 1.4]{gomez2012mecanica}
    \item Creación de un modelo que simule las medidas de nuestros sensores dados unos parámetros y un tiempo. Inicialmente incorporará giroscopio, acelerometro y magnetometro.
    \item Estudiar método para cuantificar errores respecto al modelo.
    \item Desarrollo de un integrador de las medidas del giroscopio empleando quaterniones. (Sin ruido y calibrado).\cite{AshwinNarayan2017Sep}
    \item Ecuación para obtención de orientación mediante acelerómetro + magnetómetro.
    \item Estudio de errores de cada sensor(gir., acc., mag.).
    \item Desarrollo de un filtro de kalman extendido simple tomando el giroscopio como medidas externa y ajustando error con acc, y mag.\cite{BibEntry2024Oct}
    \item Incorporar medidas de giroscopio internamente.
    \item Estudiar e incorporar efectos de calibración a modelo.
    \item Estudiar distribución normalizada de medidas GNSS.
    \item Estudiar error del barómetro.
    \item Añadir GNSS y barómetro al modelo.
    \item Ecuación de fusión de sensores.\cite{Becker_2023}
\end{enumerate}

\subsection{Implementación}

Durante esta fase se irán implementado los resultados de la fase de investigación. Inicialmente bajo simulación y posteriormente en el hardware elegido.

\begin{enumerate}
    \item Crear un módulo que incorpore el modelo de simulación.
        \begin{itemize}
            \item  Debe exponer:
            \begin{itemize}
                \item Función obtención de medidas de giroscopio
                \item Función obtención de medidas de acelerómetro
                \item Función obtención de medidas de magnetómetro
                \item Función con estado total del sistema
            \end{itemize}
        \end{itemize}
        
    \item Función error que compare estados del sistema. Interfaz matlab.
    \item Función tranformación de vel. ang. de Euler a quaternion.
    \item Función integración de quaterniones.
    \item Función orientación a partir de acelerómetro y magnetómetro. \cite{BibEntry2024Oct}
    \item Módulo Filtro Extendido de Kalman simple
    \item Aplicar calibraciones.
    \item Función obtención distribución normal GNSS
    \item Función obtención de medidas barométricas
    \item Incorporar cimulación de GNSS y barómetro al simulador
    \item Función de fusión de sensores.
    \item Adaptar módulos a esp-idf (FreeRTOS) \cite{espressif_idf_programming_guide}
    
    
\end{enumerate}

\subsection{Evaluación}

A lo largo del desarrollo del proyecto se realizarán ensayos tanto simulados como reales en los que se estudie la exactitud, precisión y fiabilidad de los sistemas.

Los resultados de los ensayos se compararán, simulados contra reales y ambos respecto a unas especificaciones \emph{(Aún por determinar)}.

%\subsection{Cronograma}


\section{Hardware preliminar}
\begin{itemize}
    \item ESP32 \cite{espressif_idf_programming_guide}
    \item GY-91 10 DoF
        \begin{itemize}
            \item MPU9250\cite{invensense2016mpu9250}
                \begin{itemize}
                    \item MPU6500 6 DoF
                    \item AK8963 3 DoF
                \end{itemize}
        \end{itemize}
        
        \begin{itemize}
            \item BMP280 \cite{bme_2024}
        \end{itemize}
    \item NEO-6M
\end{itemize}

%\subsection{Especificaciones}


\printbibliography

\end{document}
