%\documentclass{sty/eiiatfg}
\documentclass{sty/propuesta}
%\setmonofont{Noto Sans Mono}

\addbibresource{bib/main.bib}

%% Cambia los datos de tu TFG en este archivo

\title{Plantilla \LaTeX{} para elaborar el TFG en la Escuela de Ingeniería Industrial y Aeroespacial}
\author{David Morán Soriano}

% IE = Ingeniería Eléctrica
% IEIA = Ingeniería Electrónica Industrial y Automática
% IA = Ingeniería Aeroespacial
\grado{IE}

% Tu número de expediente puedes consultarlo en secretaría virtual
\expediente{225655}

% En caso de múltiples directores separa los nombres con \\
% Si solo hay un tutor no pongas \\
\tutor{%
    Francisco Moya Fernández\\
    Fernando Castillo García}


% A partir de aquí se trata de datos opcionales. Si algún dato no quieres que figure borra la línea o coméntala poniendo el signo % al principio

% Es muy conveniente proporcionar un medio de contacto con el autor.  El correo electrónico es probablemente el menos invasivo. No uses correos con nicknames extraños, es un documento profesional. En caso de duda usa el correo de la Universidad, pero recuerda que dejará de ser válido unas semanas después de que dejes de ser alumno.
\email{david.moran@alu.uclm.es}

% No uses tu teléfono personal, será accesible para cualquier usuario de la biblioteca
\phone{925 268 800}

% Una página web para el proyecto puede ser un requisito necesario en caso de que sea trabajo parcialmente financiado con un proyecto de I+D. Consulta a tu tutor
\homepage{https://github.com/UCLM-eiia-to/eiia_doc_tfg_latex}

% Tener un repositorio GIT (http://github.com) permite llevar un control de versiones. Tu tutor puede considerarlo esencial, habla con él
\gitrepo{https://github.com/UCLM-eiia-to/eiia_doc_tfg_latex.git}

% Pon una dirección si puede ser interesante para recibir correspondencia relacionada. No pongas tu dirección personal
\address{UCLM --- Escuela de Ingeniería Industrial y Aeroespacial\\
    Campus Universitario de la Real Fábrica de Armas}
\poblacion{Toledo}
\cpostal{45004}


\title{Propuesta de TFG}
\author{David Morán Soriano}

\begin{document}

\maketitle
\tableofcontents
\newpage

\section{Introducción}
En este documento se recogerá la propuesta inicial de un TFG para el grado de \emph{Ingeniería Aeroespacial} en la \emph{UCLM}.

Para este TFG se plantea el estudio, desarrollo e implementación de un sistema de navegación aérea de bajo coste mediante el empleo de medidas inerciales tomadas por sensores MEMS y respaldadas por posicionamiento satelital y medidas barométricas.

Con el fin de unificar toda la información adquirida por los distintos sensores se propone emplear un conjunto de Filtros de Kalman Expandidos ejecutándose en un SoC de bajo presupuesto.


\section{Objetivos}
A continuación se listarán los objetivos prelimirares principales y secundarios que se pretende explorar durante el desarrollo del TFG.
\begin{itemize}
    \item Principales
        \subitem Lectura en bruto de los sensores de forma periódica y confiable.
        \subitem Preprocesado e interpretación de lecturas.
        \subitem Registrar medidas de forma local(Tarjeta SD).
        \subitem Creación de un modelo de filtro de Kalman ajustado al sistema dinámico con el que se trabaja.
        \subitem Modelado de un sistema dinámico que permita probar el filtro de Kalman.
        \subitem Integración del filtro de Kalman y lectura de sensores en el hardware escogido.
        \subitem Evaluación de rendimiento del sistema (Latencia + precisión + consumo).

    \item Secundarios
        \subitem Grabar rutas de navegación en Tajeta SD.
        \subitem Comparar ruta respecto a medidas.
        \subitem Añadir telemetría.
        \subitem Proponer correcciones.
        \subitem Actuar conforme a las correcciones.
\end{itemize}

\section{Planificación}

Para abordar este proyecto se plantea tomar un enfoque expansivo en el que se comience alcanzando unos objetivos mínimos y, posteriormente se realice un proceso iterativo en el que se vaya ampliando y mejorando la funcionalidad del sistema.

Este enfoque requiere que se realice un estudio inicial en el que se plantee el alcance máximo del proyecto, de forma que se pueda diseñar el código para ser capaz de incorporar nueva funcionalidad sin necesidad de realizar alteraciones significativas.

Además, dado que se pretende alterar continuamente el código, serán necesarios mecanismos que permitan facilitar y automatizar la verificación de la funcionalidad de forma retroactiva.

El TFG estará dividido en 3 secciones principales, que serán las que se irán ejecutando de forma iterativa.

La primera consistirá en una fase de investigación en la que se estudie y plantee el modelo que se quiera implementar. El resultado será un modelo matemático y una propuesta breve sobre como implementarlo.

La segunda traducirá el modelo a un lenguaje de programación, comúnmente C, aunque también se podrán emplear Python y Matlab para hacer pruebas de funcionalidad o hacer pruebas más rápidamente. Como resultado se obtendrá un código que se podrá ejecutar en el hardware escogido.

Finalmente, se realizará una fase de evaluación en la que se comprobará el desempeño del modelo en un caso controlado, en el que se pueda comparar los resultados con los obtenidos, tanto en iteraciones anteriores como respecto a las especificaciones planteadas.

A continuación se desarrollará el proceso de cada fase de forma preliminar para la primera iteración.

\subsection{Investigación}
\subsection{Implementación}
\subsection{Evaluación}

\printbibliography

\end{document}
